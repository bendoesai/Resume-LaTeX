\\\\
\textbf{Minimum Requirements:}
\begin{itemize}
    \item \textbf{Programming Experience:} Through my education
    surrounding deep learning and related computational techniques,
    I have gathered 4+ years of experience in Python. Learning
    to code efficiently, legibly, and consistently has been
    critical in my overall success in both research and production.
    In addition, I have exposure to several other languages through
    classes or related co-op work, specifically C/C++ in both high-
    level and embedded environments.

    \item \textbf{Operating System Experience:} During my time
    with the DeFake Project, we developed all of our models and
    scripts on a dedicated remote SSH server running Ubuntu, 
    which I had to learn and navigate quickly in order to deliver
    results on pace with the rest of the team. This involved
    monitoring processes, navigating deep file directories,
    and using the command line for executing code. Through
    my coursework and extracurriculars, I have also had the 
    opportunity to develop on Windows and Linux-like Windows 
    systems (Cygwin, WSL). This breadth of exposure in 
    environments makes me a versatile developer in any platform 
    or research setting.

    \item \textbf{Version Control System Experience:} After my 
    first co-op experience, I began using Git and private GitHub
    repositories to store any project that required more than 
    three files. This allowed me to easily transfer projects 
    between devices, and get used to git's version control 
    ecosystem at the same time. This allowed me to carry lessons
    I had learned from my own experimentation into future work 
    experiences to accelerate my integration into new teams,
    and also push for better version control practices on teams 
    that haven't adapted them yet.

    \item \textbf{Collaboration/Teaming:} In my two-semester 
    senior design class, I took the role of Project Lead for our 
    4-person team early on. The duties of this position involved
    being the primary point of contact for our client that owns
    our project, organizing all the design reviews and check-ins 
    with our faculty guide, and keeping the team working efficiently
    and effectively. This allows me to be a more effective team 
    member and gives me the utility to step up when the opportunity 
    presents itself.

    \item \textbf{Communication:} RIT engineering focuses heavily on 
    laboratory classes, which means that students are required to 
    write reports summarizing labs in nearly every class. Doing this
    has given me years of practice in communicating ideas from 
    semiconductor processes to analog circuit design in a manner that 
    is concise and manageable. This directly transfers to research 
    and white-paper reports that I have worked on in the industry.

    \item \textbf{Education/Experience:} I plan to graduate 
    this May with a BS in Electrical Engineering and an immersion 
    in Applied Statistics. I intend to pursue higher education 
    with the end goal of obtaining a Ph.D.

\end{itemize}
\textbf{Desired Qualifications:}
\begin{itemize}
    \item \textbf{Satellite Senors and Programs} Through my senior 
    design project, I have learned about integrating and taking 
    measurements from a specialized SWIR sensor that uses a MEMS 
    light filter and a photodiode to detect magnitude of light 
    across a certain wavelength. I've also worked with complex 
    real-time datasets in the context of titanium manufacturing. 
    This has equipped me with the skills required to tackle complex, 
    real-world sensor problems, like one that may come from a 
    satellite using tools in python such as Pandas, Numpy, 
    Scikit-learn and Pytorch. It also equips me with the bare metal 
    knowledge required to interpret the mechanical and electrical
    phenomena that may come from the hardware side of sensors.
    
    \item \textbf{Scientific Algorithm Development} During my term 
    as a research assistant for the DeFake project, I had the 
    opportunity to write a modular, scalable experiment framework 
    to assess the performance of Deepfake Detection models and their
    robustness. The skills I learned in the development of that 
    framework would be easily applied in creating modular, repeatable 
    experimental code at LANL.
    
    \item \textbf{Data Processing and Engineering} In my artificial
    intelligence education and applied statistics immersion, I have 
    learned techniques for processing large amounts of data using 
    software like Python, as well as drawing statistically relevant 
    inference from the data using JMP and DOE principles. This allows 
    me to not only design effective research experiments, but also
    interpret the data in a way that is provably significant.
    
    \item \textbf{Unit and Integration Testing} Digital design via HDL 
    is a core aspect of our curriculum as electrical engineers. Digital 
    Design validation requires writing exhaustive test benches to 
    verify system functionality. The skills learned through writing 
    these scripts can be easily transferred to software unit testing.
    
    \item \textbf{Debugging Tools} While I am less familiar with GDB, 
    I have used it for our introductory C course to debug C code. I 
    have also used several other debuggers, most of the time built into 
    an IDE. I have plenty of practice debugging code without dedicated 
    debuggers as well across both high and low level languages.
    
    \item \textbf{HDF5 and Similar Data Formats} Through my work in
    data science, and specifically with the Neurotechnology Exploration 
    Team, I've had to work with many diverse file formats, like the 
    traditional CSV and JSON, as well as XDF and BRC files. While 
    I have not used HDF5 yet, I possess the skills to learn such a 
    standard as I have learned others in the past.

\end{itemize}