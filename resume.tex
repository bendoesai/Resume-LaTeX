\documentclass{resume} % Use the custom resume.cls style

\usepackage[left=0.4 in,top=0.4in,right=0.4 in,bottom=0.4in]{geometry} % Document margins
\newcommand{\tab}[1]{\hspace{.2667\textwidth}\rlap{#1}} 
\newcommand{\itab}[1]{\hspace{0em}\rlap{#1}}
\name{Ben Brown} % Your name
% You can merge both of these into a single line, if you do not have a website.
\address{+1(484) 788-3226 \\ Rochester, NY} 
\address{
\href{brown.ben.2019@gmail.com}{brown.ben.2019@gmail.com} \\
\href{https://linkedin.com/in/brownben2019}{linkedin.com/in/brownben2019}}  

\begin{document}

%----------------------------------------------------------------------------------------
%	OBJECTIVE
%----------------------------------------------------------------------------------------

\begin{rSection}{OBJECTIVE}

{Seeking an internship or co-op position in order to further my skills as an 
Electrical Engineer. Passionate in the fields of Machine Learning, 
Statistics, and Ethical AI.\@ {\bf Available Summer of 2023}
}
\end{rSection}
%----------------------------------------------------------------------------------------
%	EDUCATION SECTION
%----------------------------------------------------------------------------------------

\begin{rSection}{Education}

{\bf Bachelor of Electrical Engineering}, Rochester Institute of Technology (RIT) \hfill {Expected 2024}\\
{\bf Relevant Coursework:} Applied Statistics, Artificial Intelligence, Differential Equations, and Physics. \\
{\bf GPA:} 3.04


\end{rSection}

%----------------------------------------------------------------------------------------
% TECHINICAL STRENGTHS	
%----------------------------------------------------------------------------------------
\begin{rSection}{SKILLS}

\begin{tabular}{ @{} >{\bfseries}l @{\hspace{6ex}} l }
Software & Python, PyTorch, TensorFlow/Keras,
Jupyter, C/C++, MATLAB, Assembly (MSP430)
\\
Hardware & 
Circuit Analysis (AC/DC),
SMT/THT Soldering,
PCB Design (Altium),
Verilog/VHDL

\end{tabular}
\end{rSection}

\begin{rSection}{EXPERIENCE}

\textbf{AI Developer} \hfill Jan 2022 - Aug 2022\\
TIMET Morgantown \hfill \textit{Morgantown, PA}
 \begin{itemize}
    \itemsep-3pt {}
     \item Collaborated with several groups in producing technology that simultaneously simplifies operator’s jobs, and potentially saves the company seven figures annually
     \item Employed several Time Series prediction technologies, including fundamentals (RNN, LSTM, GRU) and state of the art (SCINet, FEDformer)
    \item Programmed with set standards and proper documentation to ensure proper readability by future coops and engineers
    \item Designed custom genetic algorithm to optimize XGBoost hyperparameters in production use
 \end{itemize}
 
\textbf{Podcast Manager} \hfill Oct 2019 - Jan 2022\\
Reporter Magazine \hfill \textit{Rochester, NY}
 \begin{itemize}
    \itemsep-3pt {} 
     \item Lead an interdisciplinary team of creators to produce engaging content regularly
     \item Execute self-driven projects where creativity is encouraged
    \item Complete projects with equipment including Adobe Premiere, Davinci Resolve, Audacity, and many types of camera and audio equipment 
 \end{itemize}

\end{rSection} 

%----------------------------------------------------------------------------------------
%	WORK EXPERIENCE SECTION
%----------------------------------------------------------------------------------------

\begin{rSection}{PROJECTS}
\vspace{-1.25em}
\item \textbf{Dream Presenter (RIT NXT)} {Use multiple EEG datasets to produce a GAN/diffusion-style architecture that can generate images from dreams based on Sleeping State neural activity. Scheduled for introductory showcase at this year's Imagine RIT event.}
\item \textbf{RISC Processor} {Constructed and debugged a working RISC processor in Verilog that could read and execute assembly code.}
\end{rSection} 

%----------------------------------------------------------------------------------------

%----------------------------------------------------------------------------------------
\begin{rSection}{Leadership} 
\textbf{RIT IEEE (STB20351 - Chair)} \hfill Sep 2020 - Jan 2022
 \begin{itemize}
    \itemsep-3pt {} 
     \item Serve as Chair of RIT Student Branch in order to restructure the club for success
     \item Generate publicity, as well as complete other administrative tasks
     \item Hold weekly meetings to ensure clarity and vision is shared by all Executive board
     \item Work on club constitution with the rest of the Executive Board

 \end{itemize}


\end{rSection}


\end{document}